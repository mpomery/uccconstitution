% !TEX TS-program = pdflatex
% !TEX encoding = UTF-8 Unicode

\documentclass[11pt]{article} % use larger type; default would be 10pt

\usepackage[utf8]{inputenc} % set input encoding (not needed with XeLaTeX)
\usepackage[a4paper,margin=1cm,footskip=.5cm]{geometry}

\usepackage[usenames,dvipsnames]{color} % Colors!

%%% Examples of Article customizations
% These packages are optional, depending whether you want the features they provide.
% See the LaTeX Companion or other references for full information.

%%% PAGE DIMENSIONS
%\usepackage[margin=0.5in,headsep=0.25in,footskip=0.25in]{geometry}
\usepackage{geometry} % to change the page dimensions
\geometry{a4paper} % or letterpaper (US) or a5paper or....
% \geometry{margin=2in} % for example, change the margins to 2 inches all round
% \geometry{landscape} % set up the page for landscape
%   read geometry.pdf for detailed page layout information

\usepackage{graphicx} % support the \includegraphics command and options

% \usepackage[parfill]{parskip} % Activate to begin paragraphs with an empty line rather than an indent

%%% PACKAGES
\usepackage{booktabs} % for much better looking tables
\usepackage{array} % for better arrays (eg matrices) in maths
\usepackage{paralist} % very flexible & customisable lists (eg. enumerate/itemize, etc.)
\usepackage{verbatim} % adds environment for commenting out blocks of text & for better verbatim
\usepackage{subfig} % make it possible to include more than one captioned figure/table in a single float
% These packages are all incorporated in the memoir class to one degree or another...

%%% SECTION TITLE APPEARANCE
\usepackage{sectsty}
\allsectionsfont{\sffamily\mdseries\upshape} % (See the fntguide.pdf for font help)
% (This matches ConTeXt defaults)

%%% ToC (table of contents) APPEARANCE
\usepackage[nottoc,notlof,notlot]{tocbibind} % Put the bibliography in the ToC
\usepackage[titles,subfigure]{tocloft} % Alter the style of the Table of Contents

\usepackage{fullpage} % margins
\usepackage{pdfpages} % include PDFs!
\usepackage{wrapfig} % Figure Wrapping
\usepackage{setspace}
\usepackage{titlesec}

\renewcommand{\cftdot}{} % Remove dots from ToC
%\setlength{\cftbeforechapskip}{.2ex}
%\setlength{\cftbeforesecskip}{.1ex}

\renewcommand{\cftsecfont}{\rmfamily\mdseries\upshape}
\renewcommand{\cftsecpagefont}{\rmfamily\mdseries\upshape} % No bold!


%\pagenumbering{gobble} % Remove page numbering
\addtocontents{toc}{\cftpagenumbersoff{section}}
\addtocontents{toc}{\cftpagenumbersoff{subsection}}


\usepackage{titling} % For the subtitle

\setcounter{tocdepth}{4}
\makeatletter
\renewcommand\tableofcontents{%
    \@starttoc{toc}%
}
\makeatother

\newcommand{\subtitle}[1]{
	\posttitle{
		\par\end{center}
		\begin{center}\large#1\end{center}
		\vskip0.5em
	}
}

\makeatletter
\renewcommand\thesection{\@arabic\c@section.}
\renewcommand\thesubsection{\@arabic\c@subsection.}
\makeatother

\let \savenumberline \numberline
\def \numberline#1{\savenumberline{#1}}

%%% END Article customizations
\usepackage{fancyhdr}
\pagestyle{fancy}
\renewcommand{\headrulewidth}{0pt}
\lhead{}
\chead{}
\rhead{}
\lfoot{}
\cfoot{}
\rfoot{\thepage}

\begin{document}

\noindent
\begin{center}
\textbf{\LARGE{University Computer Club Inc.}}\\
\Large{Constitution}
\end{center}

\noindent
\small{(Adopted on 21-Sep-1974)\\
(Amended on 4-Jun-1980, 16-Feb-1996, 23-Jan-1998, 6-Oct-1999, 16-Dec-2010, 8-Mar-2011 for incorporation, XX-February-2015)\\}

%\tableofcontents
\begingroup
	\let\cleardoublepage\clearpage
	\def\addvspace#1{}
	\tableofcontents
\endgroup


\section{Name}
\begin{description}
	\item The name of this organisation shall be: \hfill \\
		\large{\textbf{The University Computer Club Inc.}}
\end{description}

\section{Definitions}
Throughout this Constitution, unless the context otherwise indicates, the following terms shall have the meanings set against them:
\begin{description}
	\item[Club] \hfill \\
		The University Computer Club Inc.
	\item[financial year] \hfill \\
		A period commencing on the date of incorporation of the Club and ending on 31 December; and thereafter each period commencing 1 January and ending on 31 December in the same year
	\item[General Meeting] \hfill \\
		Annual General Meeting or other Ordinary or Special General Meeting.
	\item[special resolution] \hfill \\
		A resolution is a special resolution if it is passed by a majority of not less than threefourths of the members of the Club who are entitled under the rules of the Club to vote and vote in person
	\item[the Act] \hfill \\
		The \emph{Associations Incorporation Act 1987}
	\item[the Commissioner] \hfill \\
		The Commissioner for Consumer Protection exercising powers under the Act
	\item[University] \hfill \\
		The University of Western Australia
\end{description}

\section{Objects}
The objects of the Club shall be as follows:
\begin{enumerate}
	\item To be an organised association of students attending The University of Western Australia, and supporters, for the advancement of computer science and technologies, both at the University and in the broader community.
	\item To co-operate with all bodies of similar aims.
	\item The property and income of the Club shall be applied solely towards the promotion of the objects of the Club and no part of that property or income may be paid or otherwise distributed, directly or indirectly, to any person or body, except in good faith in the promotion of those objects.
\end{enumerate}

\section{Executive}
The Executive of the Club shall consist of the following members:
\begin{enumerate}
	\item The President
	{\color{red} \item The Vice-President}
	{\color{ForestGreen} \item The Education Vice-President}
	{\color{ForestGreen} \item The Social Vice-President}
	\item The Secretary
	\item The Treasurer
\end{enumerate}

\section{Committee}
The Club shall be governed by the Committee.\\
The Committee of the Club shall consist of the following members:
\begin{enumerate}
	\item The President
	{\color{red} \item The Vice-President}
	{\color{ForestGreen} \item The Education Vice-President}
	{\color{ForestGreen} \item The Social Vice-President}
	\item The Secretary
	\item The Treasurer
	{\color{ForestGreen} \item The Education Officer}
	{\color{ForestGreen} \item The Social Officer}
	{\color{red} \item Three Ordinary Committee Members}
	{\color{ForestGreen} \item Two Ordinary Committee Members}
	\item The First Year Representative
\end{enumerate}

\section{Membership}
\begin{enumerate}
	\item Membership is open to any person who shares the aims of the Club and agrees to comply with the Constitution and rules of the Club.
	\item A subscription for ordinary membership may be payable to the Club if the Committee so desires.
	\item All members must comply with all provisions relating to affiliated societies included in the Guild Statute Book.
\end{enumerate}

\section{Honorary Life Membership}
The Club in General Meeting may by a two-third majority of those present and voting confer Honorary Life Membership upon any member who has performed outstanding service to the Club.

\section{Patron and Vice Patron}
The Club at the General Meeting may elect a Patron and Vice Patron who if they indicate their willingness so to act shall hold office until the succeeding Annual General Meeting.

\section{Election}
\begin{enumerate}
	\item The following members of the Committee shall be elected at the Annual General Meeting:
		\begin{enumerate}[1.]
			\item The President
			{\color{red} \item The Vice-President}
			{\color{ForestGreen} \item The Education Vice-President}
			{\color{ForestGreen} \item The Social Vice-President}
			\item The Secretary
			\item The Treasurer
			{\color{ForestGreen} \item The Education Officer}
			{\color{ForestGreen} \item The Social Officer}
			{\color{red} \item Three Ordinary Committee Members}
			{\color{ForestGreen} \item Two Ordinary Committee Members}
			\item The First Year Representative
		\end{enumerate}
	\item The Secretary shall call for nominations for the above positions at least ten days before the date of the Annual General Meeting. The nominations shall close at that meeting.
	\begin{enumerate}[1.]
			\item Any member who has been nominated for a position and is unable to attend the Annual General Meeting must notify the Secretary of his or her acceptance of said position, in writing, before the close of nominations.
		\end{enumerate}
% Is anyone a financial member of the Guild?
	\item Only students at The University of Western Australia who are financial members of the Student Guild of The University of Western Australia may hold positions on the Executive.
	\item The First Year Representative must be a student at The University of Western Australia in first year for the first time and an ordinary member of the Club.
	\item Any members of the Club may hold a position of Ordinary Committee Member.
	{\color{ForestGreen} \item Any member of the Club may hold the position of Education Officer.}
	{\color{ForestGreen} \item Any member of the Club may hold the position of Social Officer.}
	\item Any members of the Club may vote for the positions of Committee members.
\end{enumerate}
{\color{Cyan}NOTE:
The changes to committee are splitting the Vice President Role into Education Vice President and Social Vice President, and splitting one of the Ordinary Committee Member positions into Social Officer and Education Officer.
The new officer positions have the same requirements as Ordinary Committee Members.
}

\section{Duration of Office}
The members of the Committee shall remain in office until the close of the succeeding Annual General Meeting, except as elsewhere provided in the Constitution.

\section{Duties of the Committee}
The Committee shall be responsible to the Club in General Meeting for giving effect to the objects of the Club.

\section{Exercise of Powers}
The Committee shall only exercise its powers at a properly convened meeting of the Committee, except as elsewhere provided in the Constitution.

\section{Powers of the Club}
The Club may do all things necessary or convenient for carrying out its objects and purposes, and in particular, may:
\begin{enumerate}
	\item acquire, hold, deal with, and dispose of any real or personal property;
	\item open and operate bank accounts;
	\item invest its money:
	\begin{enumerate}[1.]
			\item in any security in which trust monies may lawfully be invested; or
			\item in any manner authorised by the rules of the Club;
		\end{enumerate}
	\item borrow money or incur overdrafts provided always that the total long term liabilities of the Club shall not exceed {\color{red}one} {\color{ForestGreen}two} hundred dollars, unless otherwise approved by a General Meeting;
	\item give such security for the discharge of liabilities incurred by the Club as the Club thinks fit;
	\item appoint agents to transact any business of the Club on its behalf;
	\item enter into any other contract it considers necessary or desirable; and
	\item act as trustee and accept and hold real and personal property upon trust, but does not have the power to do any act or thing as a trustee that, if done otherwise than as a trustee, would contravene the rules of the Club.
\end{enumerate}

{\color{Cyan}NOTE:
There has been a lot of debate about if the long term liabilities should change. If we assume that they were set around 1998, then adjusting for inflation, that number should be around \$156, so upping it to \$200 is reasonable.
}

\section{Common Seal of the Club}
\begin{enumerate}
	\item The Club must have a common seal on which its corporate name appears in legible characters.
	\item The common seal of the Club must not be used without the express authority of the Committee and every use of that common seal must be recorded by the Secretary.
	\item The affixing of the common seal of the Club must be witnessed by any two Executive members.
	\item The common seal of the Club must be kept in the custody of the Secretary or of such other person as the Committee from time to time decides.
\end{enumerate}

\section{Proceedings of Committee}
\subsection{Ordinary Meetings}
The Committee shall meet at least once a month at such times and places as the President or the Committee may determine.
\subsection{Notice of Meetings}
The Secretary shall cause all members of the Committee to receive written or email notice of any meeting of the Committee at least four days before the date of the meeting.
\subsection{Special Meetings}
\begin{enumerate}
	\item The Secretary shall forthwith call a meeting of the Committee upon receiving a written requisition from at least three members thereof; and such special meetings shall be held not later than seven days immediately following the receipt of such requisition.
	\item If the Secretary fails to announce a meeting within 3 days, any one of the members signing the requisition may do so and must give the same notice as required of the Secretary.
	\item Any business set out in the requisition shall have priority over all other business.
\end{enumerate}

\subsection{Quorum}
The quorum for a meeting of the Committee shall be {\color{red}four} {\color{ForestGreen}five} members of the Committee at least {\color{red}two} {\color{ForestGreen}three} of which are members of the Executive.\\
{\color{Cyan}NOTE:
There are several places in the constitution that have been changed to adjust for an increase in committee size.
}

\subsection{Declaration of Pecuniary Interest}
A Committee member having any direct or indirect pecuniary interest in a contract, or proposed contract, made by, or in the contemplation of, the Committee (except if that pecuniary interest exists only by virtue of the fact that the member of the Committee is a member of a class of persons for whose benefit the Club is established), must:
\begin{enumerate}
	\item as soon as he or she becomes aware of that interest, disclose the nature and extent of his or her interest to the Committee; and
	\item not take part in any deliberations or decision of the Committee with respect to that contract.\\
\texttt{Sub-rule 5.1 does not apply with respect to a pecuniary interest that exists only by virtue of the fact that the member of the Committee is an employee of the Club.}
\end{enumerate}

\subsection{Voting}
\begin{enumerate}
	\item Only members of the Committee are entitled to vote at a meeting of the Committee.
	% How this next point relates to Door Group
	\item Except where stated otherwise in the Constitution, a motion shall be passed if a majority of those present and eligible to vote cast their vote in favour of the motion.
	% Abstaining?
\end{enumerate}

\subsection{Rules of Debate}
Any member of the Committee may demand at any meeting of the Committee that the meeting be conducted in accordance with the current edition of Robert’s Rules of Order Newly Revised in all cases to which they are applicable and in which they are not inconsistent with this Constitution or policy made thereunder.

\section{Duties of Committee Members}
\subsection{President}
In addition to any other provisions set out elsewhere in this Constitution, or any regulation or policy made thereunder, it shall be the duty of the President:
\begin{enumerate}
	\item To conduct the relations of the Club with other organisations and to the general public subject to the authority of the Committee;
	\item To co-ordinate and supervise the work of the {\color{red}Vice-President,} {\color{ForestGreen}Education Vice-President, Social Vice-President,} Secretary, Treasurer, and other officers of the Club subject to the authority of the Committee;
	\item To generally carry out the policy of the Club; and
	\item To generally carry out the instructions and decisions of the Committee.
\end{enumerate}
{\color{Cyan}NOTE:
This change simply reflects the splitting of the vice president role.
}

{\color{red} \subsection{Vice President}
In addition to any other provisions set out elsewhere in this Constitution, or any regulation or policy made thereunder, it shall be the duty of the Vice-President:
\begin{enumerate}
	\item To assist in the President's duties;
	\item To carry out Presidential duties during the absence of the President, during which time the Vice-President shall be deemed to be the President; and
	\item To generally carry out the instructions of the Committee.
\end{enumerate}}

{\color{ForestGreen} \subsection{Education Vice-President}
In addition to any other provisions set out elsewhere in this Constitution, or any regulation or policy made thereunder, it shall be the duty of the Education Vice-President:
\begin{enumerate}
	\item To assist in the President's duties;
	\item To conduct the relations of the Club with other organisations and to the general public subject to the authority of the Committee when they pertain to education;
	\item To oversee and encourage the running of educational events by the club; and
	\item To generally carry out the instructions of the Committee.
\end{enumerate}}

{\color{ForestGreen} \subsection{Social Vice-President}
In addition to any other provisions set out elsewhere in this Constitution, or any regulation or policy made thereunder, it shall be the duty of the Social Vice-President:
\begin{enumerate}
	\item To assist in the President's duties;
	\item To conduct the relations of the Club with other organisations and to the general public subject to the authority of the Committee when they pertain to social matters;
	\item To oversee and encourage the running of social events by the club; and
	\item To generally carry out the instructions of the Committee.
\end{enumerate}}

{\color{Cyan}NOTE:
The aim of the Education Vice President and Social Vice President is to split the role into two manageable parts that are better defined.
In the event were the president is absent, the acting president will be decided from a vote by committee.
}

\subsection{Secretary}
In addition to any other provisions set out elsewhere in this Constitution, or any regulation or policy made thereunder, it shall be the duty of the Secretary:
\begin{enumerate}
%%% Incorporations Act may change to remove this requirement

	\item To maintain an up to date register of the members of the Club and {\color{red}their postal or residential addresses} {\color{ForestGreen}the information required by the Act} and, upon the request of a member of the Club, make the register available for the inspection of the member;
	\begin{enumerate}[1.]
			\item A member may make a copy of or take an extract from the register but shall have no right to remove the register for that purpose.
		\end{enumerate}
	\item To carry out the administrative work of the Club for which the Committee does not appoint or elect an officer;
	\item To record the proceedings of all General Meetings and meetings of the Committee and make these minutes available in an easily accessible portion of the Club online presence;
	\item To conduct and keep copies of all correspondence to the Club;
	\item To generally carry out the instructions and decisions of the Committee relating to the administration of the Club;
	\item To ensure that an accurate copy of the Club constitution is maintained in an easily accessible portion of the Club online presence; and
	\item To unless the members resolve otherwise at a general meeting, have custody of all books, documents, records and registers of the Club, other than those required to be kept and maintained by, or in the custody of, the Treasurer.
\end{enumerate}
{\color{Cyan}NOTE:
Currently the Act (s 27) requires the collection of a residential or postal address for all members, however there are pending changes to the Act that may mean we can collect only an email address instead.
The change in wording here should allow UCC to change the collected information as soon as the Act is updated.\\
}


\subsection{Treasurer}
In addition to any other provisions set out elsewhere in this Constitution, or any regulation or policy made thereunder, it shall be the duty of the Treasurer:
\begin{enumerate}
	\item To be responsible for the receipt of all moneys paid to or received by, or by him or her on behalf of, the Club and issue receipts for those moneys in the name of the Club;
	\item To keep accounting records as correctly record and explain the financial transactions and financial position of the Club, in such a manner that permits convenient and proper auditing;
	\item To render an account at each meeting of the Committee of all receipts and payments since the previous meeting;
	\item To render an account of the petty cash at each meeting of the Committee;
	\item To furnish the Committee with such other accounts and information relating to the finances of the Club as the Committee may require;
	\item To present a current financial statement showing all receipts and payments at each Ordinary General Meeting and by motion of Committee, an auditor's report at the Annual General Meeting;
	\item To unless the members resolve otherwise at a general meeting, have custody of all securities, books and documents of a financial nature and accounting records of the Club; and
	\item To generally carry out the instructions of the Committee relating to the property and finances of the Club.
\end{enumerate}

{\color{ForestGreen} \subsection{Education Officer}
In addition to any other provisions set out elsewhere in this Constitution, or any regulation or policy made thereunder, it shall be the duty of the Education Officer:
\begin{enumerate}
	\item To assist in the Education Vice-President's duties;
	\item To facilitate the running of educational events; and
	\item To generally carry out the instructions of the Committee.
\end{enumerate}}

{\color{ForestGreen} \subsection{Social Officer}
In addition to any other provisions set out elsewhere in this Constitution, or any regulation or policy made thereunder, it shall be the duty of the Social Officer:
\begin{enumerate}
	\item To assist in the Social Vice-President's duties;
	\item To facilitate the running of social events; and
	\item To generally carry out the instructions of the Committee.
\end{enumerate}}

{\color{Cyan}NOTE:
The Education and Social Officer roles replace one of the Ordinary Committee Member positions.
These positions compliment the new Vice President roles to help ensure no one burns out.
}

\subsection{Ordinary Committee Members}
In addition to any other provisions set out in this Constitution, or in any regulation or policy made thereunder, it shall be the duty of the Ordinary Committee Members:
\begin{enumerate}
	\item To assist the other Committee Members in the executions of their duties; and
	\item To bear any responsibilities that the Committee shall from time to time decide.
\end{enumerate}

\subsection{First Year Representative}
In addition to any other provisions set out in this Constitution, or in any regulation or policy made thereunder, it shall be the duty of the First Year Representative:
% Gather feedback from the freshers about events they want to see happen
% Help to organise events for freshers and publicize them
\begin{enumerate}
	\item To represent the interests of members of the Club who are in first year University or who have not previously been members of the Club.
	{\color{ForestGreen}\item To assist the other Committee Members in the executions of their duties, especially in the case were they relate to new members; and
	\item To bear any responsibilities that the Committee shall from time to time decide.}
\end{enumerate}
{\color{Cyan}NOTE:
The change here is to define the First Year Representative position as something more than "The Freshers Are Happy".
}

\section{Vacancies}
\subsection{Resignation}
Any officer of the Club may resign from a position by giving written notification to the Secretary at least fourteen days before that Officer's resignation takes effect.
\subsection{Absence}
The Committee shall declare vacant the position of any member of the Committee if that member has been absent from two consecutive meetings of the Committee without giving a satisfactory explanation for that absence.
\subsection{Election}
If any position on the Committee becomes vacant then the Committee shall call a General Meeting for the election of a member to that position within four weeks of that position becoming vacant.
{\color{ForestGreen}If the vacant position is an executive role, the committee may choose to elect a member of the committee to temporarily fill that role until the General Meeting is held.}

\section{General Meetings}
\subsection{Ordinary General Meetings}
There shall be at least one General Meeting in each year, namely the Annual General Meeting, and any Ordinary General Meetings held as the Committee shall determine. The annual general meeting must be held in every calendar year within 4 months after the end of the Club's financial year or such longer period as may in a particular case be allowed by the Commissioner, except for the first annual general meeting which may be held at any time within 18 months after incorporation.
\subsection{Notice of Meeting}
The Secretary shall make all reasonable attempts to notify all members using at least a notice on the club notice-board for ten days preceding the meeting and an email sent to all addresses on the membership register at least ten days before the date of the meeting.
\subsection{Special General Meeting}
The Committee may at any time, without the need for a Committee meeting, call a Special General Meeting of the Club by the written agreement of at least {\color{red}4} {\color{ForestGreen}5} Committee members, at least {\color{red}one} {\color{ForestGreen}two} of whom is a member of the Executive. This meeting must be notified by the Secretary as with all other General Meetings, or if the Secretary has resigned or is unavailable by any member of the Committee with the same notice requirements of the Secretary.\\
{\color{Cyan}NOTE:
Another change of numbers.
}

\subsection{General Meeting Called by Ordinary Members}
\begin{enumerate}
	\item The Secretary shall forthwith call a Special General Meeting upon receiving a written requisition from at least ten ordinary members of the Club and such meetings shall be held with appropriate notice requirement and not later than 17 days immediately following the receipt of such requisition.
	\item If the Secretary fails to announce a meeting within 7 days any one of the members signing the requisition may do so and must make all reasonable attempts to give the same notice as required of the Secretary.
	\item Any business set out in the requisition shall have priority over all other business.
\end{enumerate}

\subsection{Powers of a General Meeting}
A General Meeting may exercise the same powers as the Committee except for those powers exclusively reserved for the Committee under this Constitution.
\subsection{Agenda}
The Secretary shall post the agenda for any General Meeting on the notice-boards alongside the notice of meeting at least four days before the date of the meeting.

\subsection{Notice of Motion}
At least four days before the meeting, topics intended to be discussed at the meeting shall be posted on the notice-board alongside the notice of meeting.

\subsection{Quorum}
The quorum of a General Meeting shall be twenty ordinary members present in person.

\subsection{Rules of Debate}
A General Meeting shall be conducted in accordance with the current edition of Robert’s Rules of Order Newly Revised in all cases to which they are applicable and in which they are not inconsistent with this Constitution or policy made thereunder.

\subsection{Chair}
% Needs rewriting
{\color{red}The President shall have the right to take the chair at any General meeting or meetings of the Committee. If the President is absent or does not wish to exercise his right at any meeting, that right shall devolve upon the Vice-President being absent or not wishing to exercise the above right, that meeting shall elect its own Chair.}\\
{\color{ForestGreen}The President shall have the right to take the chair at any General meeting or meetings of the Committee. If the President is absent or does not wish to exercise his right at any meeting, that meeting shall elect its own Chair.}\\
{\color{Cyan}NOTE:
The wording has also been redone to make it clearer. The meeting can elect it's own chair if the President does not take the position.
}

\subsection{Voting}
\begin{enumerate}
	\item Except where stated otherwise in the constitution, all members of the Club are entitled to vote at a General Meeting.
	\item Except where stated otherwise in the constitution, a motion shall be passed if a majority of those present and eligible to vote cast their vote in favour of the motion.
	\item Postal voting or proxy voting is not permitted.
\end{enumerate}

\subsection{Procedure for Motions}
\begin{enumerate}
	\item Once a motion is tabled, the motion shall be open to general discussion and amendments.
	\item After general discussion, a call for votes will be made by the Chair of the meeting.
	\item Any member at the meeting may call for the votes to be by secret ballot.
\end{enumerate}

\section{Delegates}
The Committee shall have the power, after its re-constitution, to appoint from amongst its members a Delegate and Deputy-Delegate to any body with which it may choose to affiliate.

\section{Removal From Office}
\begin{enumerate}
	\item The Committee may request any member of the Committee, member of any subcommittee, officer, representative or delegate of the Club to resign from that position if a person has failed to perform satisfactorily the normal duties of the position.
	\item The person concerned shall be informed of this request at least seven days before a meeting of the Committee where that person may speak on their own behalf.
	\item Any such request to resign shall require such a motion which will only be passed if a two-thirds majority of those voting have voted in favour of the motion. The request to resign shall be delivered in writing to the person concerned forthwith.
	\item If a person has not resigned within seven days of the request the Committee may declare that position vacant.
	\item Any person who has had their former position declared vacant shall have the right to appeal to a General Meeting for their reinstatement.
\end{enumerate}

\section{Cancellation of Membership}
\begin{enumerate}
\item The Club may cancel the membership of any member of the Club, should that member have acted in a manner contrary to the best interests of the Club.
\item A two-thirds majority of a committee meeting may cancel a membership by passing a motion to that effect.
\item The cancellation of membership shall be delivered in writing to the person concerned forthwith.
\item Any person who has their membership cancelled shall have the right to appeal to a General Meeting for the reinstatement of their membership, except where that person is a first-time member of the Club and has been a member for less than one month or has had their membership cancelled by Committee more than once.
\item A person who has had their membership cancelled may not reapply for membership for the period of one year.
\end{enumerate}

\section{Finance}
\begin{enumerate}
	\item All money due and payable to the Club shall be received by the Treasurer who shall lodge it without undue delay in the account of the Club.
	\item Except as otherwise provided for in the Constitution, or in a policy, any two of the Executive shall be empowered jointly to sign cheques and forms of authority for the withdrawal of any money standing in the credit of the Club in the account of the Club.
	\item The Committee may by a two-thirds majority elect a single delegate from the Committee who may sign in lieu of one of the required Executive signatures on a cheque.
	\item No payment shall be made on behalf of or in the name of the Club unless it has been authorised by the Committee.
	\item The Committee may, outside a Committee meeting, agree to a payment or purchase by written consent of {\color{red}five} {\color{ForestGreen}six} Committee members including at least {\color{red}two} {\color{ForestGreen}three} Executive members of the Committee. Such written consent must be tabled and minuted at the next meeting of the Committee.
\end{enumerate}
{\color{Cyan}NOTE:
Another change of numbers.
}

\section{Policy}
\begin{enumerate}
	\item The Club in General Meeting may make policy on any matter relating to the conduct of the affairs of the Club provided that policy is not inconsistent with or repugnant to this Constitution or the Act.
	\item Creation, modification, or removal of policy shall require written notice and shall only be passed by special resolution at a General Meeting.
	\item The policy of the Club shall form an appendix to this Constitution and be made available with it.
\end{enumerate}

\section{Interpretation of the Constitution}
The Committee shall be the sole authority for interpreting the meaning of any of the provisions contained in this Constitution or in any regulation of the Club policy made thereunder.

\section{Amendments to the Constitution}
\begin{enumerate}
	\item The Club in a General Meeting may amend this Constitution.
	\item Any motion proposing to amend this Constitution shall require a written notice of motion and any such motion shall only be passed by special resolution.
	\item Within one month of the passing of a special resolution altering its Constitution, or such further time as the Commissioner may in a particular case allow (on written application by the Club), the Club must lodge with the Commissioner notice of the special resolution setting out particulars of the alteration together with a certificate given by a member of the Committee certifying that the resolution was duly passed as a special resolution and that the Constitution of the Club as so altered conforms to the requirements of the Act.
	\item An alteration of the Constitution of the Club does not take effect until sub-rule 25.3 is complied with.
\end{enumerate}

\section{Interim Measures}
% Keep in "<bob> we deliberately left interim measures in there from the original constitution for posterity"
The Executive shall have the power to suspend such provisions of this Constitution until the 15th April, 1975 as may be expedient to the formation of a Club.\\
{\color{Cyan}NOTE:
bob: we deliberately left interim measures in there from the original constitution for posterity
}

\section{Commercial Exploitation}
No member of the Club shall use any of the resources of the Club for direct financial gain.

\section{Inspection of Constitution, records, etc.}
\begin{enumerate}
	\item The Secretary shall make available to any member of the public a copy of this Constitution for perusal.
	\item A member may at any reasonable time inspect without charge the books, documents, records and securities of the Club.
\end{enumerate}

\section{Dissolution}
If upon the winding up or dissolution of the Club there remains after satisfaction of all its debts and liabilities any property whatsoever, the same must not be paid to or distributed among the members, or former members. The surplus property must be given or transferred to another association incorporated under the Act which has similar objects and which is not carried out for the purposes of profit or gain to its individual members, and which association shall be determined by resolution of the members.

\end{document}
